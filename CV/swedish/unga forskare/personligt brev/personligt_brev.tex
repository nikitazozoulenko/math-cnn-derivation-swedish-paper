\documentclass[11pt,a4paper]{article} % Font sizes: 10, 11, or 12; paper sizes: a4paper, letterpaper, a5paper, legalpaper, executivepaper or landscape; font families: sans or roman

\usepackage[T1]{fontenc}
\usepackage[utf8]{inputenc}
\usepackage[swedish]{babel}
\usepackage[a4paper, total={6in, 8in}]{geometry}

\setlength{\parindent}{0em}
\setlength{\parskip}{1em}
\begin{document}

%----------------------------------------------------------------------------------------
%   COVER LETTER
%----------------------------------------------------------------------------------------
\begin{center}
\textbf{Personligt Brev}
\end{center}

\begin{flushright}
\textit{Nikita Zozoulenko} \\
Tyrgatan 2\\
Stockholm 114 27, Sverige \\
\today
\end{flushright}
Hej,

Mitt namn är Nikita Zozoulenko och jag studerar Teknisk Fysik på Kungliga Tekniska Högskolan (KTH). Jag har levt i Linköping hela mitt unga liv men valde att flyttade upp till Stockholm för att utmana mig själv med det jag anser är Sveriges svåraste matematiska utbildning, samtidigt som jag sätts i en ny stad där jag kan utvecklas vidare. Personlig utveckling är något som jag en stor fascination för. Jag tror att det är viktigt att leva ett balanserat liv; att ens karaktärsmässiga utveckling är lika viktig som ens akademiska utveckling. Det är enkelt att ägna en stor del av sitt liv åt att få specifika tekniska kompetenser, men det är svårare och mer eftertraktat att utveckla en större helhet och social kompetens. Egenskaper som god ledarskapsförmåga, ödmjukhet och hög social kompetens är något som jag försöker utveckla utanför mitt akademiska liv. 

Min lust för kunskap och vidare utveckling startade omkring den tiden jag började gymnasiet. I grund och botten har jag alltid varit en problemlösare med ett stort intresse för matematik, men det var inte förrän jag började Katedralskolan i Linköping som jag insåg vilken potential vi har om vi gör allt vi kan för att förbättras. På mina första två år läste jag all den matematik som gymnasiet hade att erbjuda. Sedan läste jag en kurs i naturvetenskaplig specialisering i matematik och linjär algebra på Linköping Universitet. Detta var samtidigt som jag engagerade mig i skolans och grannskolans programmeringssällskap och två politiska föreningar utanför skolan. 

Det var under denna tid mitt stora intresse för artificiell intelligens och maskininlärning uppstod. Det som fascinerade mig var hur man drar gränsen mellan verkligt och artificiellt liv. Ett personligt motto som jag följer är att om jag vill veta hur det är att jobba inom en visst fält så provar jag på att göra det nu istället för att vänta 5 år, läsa en master, och sedan inse att det inte var för mig. Därför läste jag den statistik, linjär algebra och analys som krävdes för att börja med maskininlärning på min fritid. Detta var min ingång till Utställningen Unga Forskare där jag bidrog med min maskininlärningsmodell för ansiktsigenkänning i stora folkmassor för realtidsvideo, samt forskning inom sekvensmodellering som med hjälp av mina förbättringar uppnådde bättre resultat än de bästa forskarna.  

Efter utställningen fortsatte jag min personliga utveckling på Intel Internation Science and Engineering Fair (Intel ISEF) som var en ögonöppnande upplevelse. Den gjorde så jag skapade vänner för livet, men viktigast av allt fick den mig att inse vårt potential för framtiden. Människor i samma ålder som jag hade gjort otroliga projekt som jag aldrig viste var möjliga att skapa. Tävlingsbidragen varierade mellan utvecklingen av autonoma robotar som klättrar på skyskrapor till nya metoder inom medicin för att förebygga Alzheimers sjukdom.  Själv blev jag hedrad att ta emot tre priser, varav ett var förstaplats i världen inom utvecklingen av maskininlärning och artificiell intelligens. Det har varit oerhört inspirerande att se alla projekt från Intel ISEF och de upplevelserna fortsätter att driva mig och mina projekt inom maskininlärning.  

Efter resan började jag drivas av en ökad lust för kunskap och utveckling, både karaktärsmässigt och akademiskt. Efter utställningen blev jag erbjuden att forska på KTH innan jag började studera teknisk fysik. På samma sommar blev jag anställd som forskningsingenjör på ContextVision där jag arbetade inom digital patologi för att detektera cancer på mikrometer-nivå i hematoxylin-eosin-färgningar av medicinska bilder. Under anställningen skrev jag dessutom en vetenskaplig artikel med namnet \textit{Gland Instance Segmentation Through Overlapping Contour Regions and Random Transformation Sampling}. Vid sidan av mina studier på universitetet har jag ägnat mig åt personliga projekt. Jag tror att projektarbeten är en av de bästa metoderna för inlärning av nya avancerade koncept. Det är en form av kreativt nyskapande och är grundat i verkligheten, något som man kan känna att teoretisk matematik saknar. Bland annat har jag skapat en maskininlärnings-agent som utan någon mänsklig kunskap lärt sig själv en algoritm för att lösa alla möjliga 2x2 Rubiks Kuber.  Jag har dessutom använt mig av samma matematisk teori som Google använde för att slå världsmästaren i brädspelet Go för att skapa en agent som kan slå människor i NxN tic-tac-toe, genom att endast spela mot sig själv. 

På den fri tid jag har utanför mina studier på KTH, de extra kurserna i matematik och mina projektarbeten gillar jag att vara med andra människor. Förutom artificiell intelligens och matematik gillar jag musik, körsång, träning, resor i naturen, och att träffa nya människor med nya livserfarenheter. På universitetet engagerar jag mig i den ädla konsten av studentteater (spex) och tränar minst tre gånger i veckan. Utanför skolan sjunger jag i en kör och förbereder mig för konsert i januari månad. 

Tack,

Nikita Zozoulenko


%----------------------------------------------------------------------------------------

\end{document}