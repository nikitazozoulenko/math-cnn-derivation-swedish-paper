\documentclass[11pt,a4paper]{article} % Font sizes: 10, 11, or 12; paper sizes: a4paper, letterpaper, a5paper, legalpaper, executivepaper or landscape; font families: sans or roman

\usepackage[T1]{fontenc}
\usepackage[utf8]{inputenc}
\usepackage[swedish]{babel}
\usepackage[a4paper, total={6in, 9in}]{geometry}

\setlength{\parindent}{0em}
\setlength{\parskip}{1em}
\begin{document}
\pagenumbering{gobble}

%----------------------------------------------------------------------------------------
%   COVER LETTER
%----------------------------------------------------------------------------------------
\begin{flushright}
\textit{Nikita Zozoulenko} \\
Tyrgatan 2\\
Stockholm 114 27, Sweden \\
\today
\end{flushright}
Greetings,

My name is Nikita Zozoulenko and I am currently studying engineering physics (swedish: \textit{teknisk fysik}) at KTH Royal Institute of Technology. I like to think of myself as a problem solver at heart, always searching for new challenging problems and personal projects to further develop myself. Alongside my university studies I have been working on a number of machine learning projects. Some of these include state-of-the-art computer vision detection and segmentation models, a reinforcement learning agent learning to solve any $2 \times 2$ Rubiks Cube by learning to predict the outcome of a Monte Carlo Tree Search, and a single-threaded PyTorch implementation of Google's self-play reinforcement learning agent for zero-sum games.

My interest for mathematics and machine learning developed during gymnasium where I studied mathematics one year ahead of my class and then read linear algebra at Linköping University. During my free time I learned the basics of machine learning when I started working on a number of small personal projects: face detection for big crowds, real-time style transfer, automatic image annotation, and implementing convolutional networks from scratch in NumPy. 

In March 2018 I entered the national science and engineering competition \textit{Utställningen Unga Forskare} (english: Exhibition Young Scientists) where I managed to win the first place award of representing Sweden at the Intel International Science and Engineering Fair (ISEF), where the best 1800 students in the world competed. It was an eye opening experience where I made friends for life and got to see what was possible to achieve as a student my own age, from finding new ways to treat Alzheimer's Disease to autonomous skyscraper cleaning robots. I also had the honor of winning three prizes, most notably first prize in the world in machine learning and AI from AAAI for my project consisting of dense face detection and improving temporal convolutional networks for new state-of-the-art results. My experiences from Intel ISEF continue to drive my passion for mathematics, engineering, and personal development to this day.

After my accomplishments at Intel ISEF I was given an offer to work at ContextVision AB as a machine learning engineer. I worked in the digital pathology team on a project consisting of detecting cancer at micrometer level in haematoxylin and eosin stained medical images. During my employment I also wrote a paper titled \textit{Gland Instance Segmentation Through Overlapping Contour Regions and Random Transformation Sampling}.

I believe that a persons character development is just as important as their academic or professional development. That is why I like to spend the little free time I have after my studies with other people. I think traits like good leadership, communication and social skills are essential in life. My hobbies include sports, music, singing in choirs, traveling nature and meeting new people of different backgrounds. I try to maintain a training schedule of working out four times a week, and I am currently preparing for a concert in January with my choir.



%----------------------------------------------------------------------------------------

\end{document}